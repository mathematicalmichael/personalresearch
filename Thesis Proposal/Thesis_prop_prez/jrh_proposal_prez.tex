\documentclass[11pt]{beamer}


\usepackage{amssymb, amsmath, graphicx, caption, enumerate}
\graphicspath{ {images/} }
\usepackage{amsthm}
\usepackage{xargs}
\usepackage{scalerel}




\newcommand{\N}{\mathbb{N}}
\newcommand{\Z}{\mathbb{Z}}
\newcommand{\R}{\mathbb{R}}
\newcommand{\K}{\mathbb{K}}
\newcommand{\C}{\mathbb{C}}
\newcommand{\ds}{\displaystyle}
\newcommand{\op}[1]{\left(#1\right)}
\newcommand{\cp}[1]{\left[#1\right]}
\newcommand{\av}[1]{\left| #1\right|}
\newcommand{\st}[1]{\left\{#1\right\}}


\usepackage[colorinlistoftodos,prependcaption,textsize=tiny]{todonotes}
\newcommandx{\question}[2][1=]{\todo[linecolor=red,backgroundcolor=red!25,bordercolor=red,#1]{#2}}
\newcommandx{\change}[2][1=]{\todo[linecolor=blue,backgroundcolor=blue!25,bordercolor=blue,#1]{#2}}
\newcommandx{\add}[2][1=]{\todo[linecolor=OliveGreen,backgroundcolor=OliveGreen!25,bordercolor=OliveGreen,#1]{#2}}
\newcommandx{\improve}[2][1=]{\todo[linecolor=Plum,backgroundcolor=Plum!25,bordercolor=Plum,#1]{#2}}
\newcommandx{\thiswillnotshow}[2][1=]{\todo[disable,#1]{#2}}
\newcommandx{\remove}[2][1=]{\todo[linecolor=yelllow,backgroundcolor=yellow!10,bordercolor=red,#1]{#2}}


\newcommand\reallywidehat[1]{\arraycolsep=0pt\relax%
\begin{array}{c}
\stretchto{
  \scaleto{
    \scalerel*[\widthof{\ensuremath{#1}}]{\kern-.5pt\bigwedge\kern-.5pt}
    {\rule[-\textheight/2]{1ex}{\textheight}} %WIDTH-LIMITED BIG WEDGE
  }{\textheight} % 
}{0.5ex}\\           % THIS SQUEEZES THE WEDGE TO 0.5ex HEIGHT
#1\\                 % THIS STACKS THE WEDGE ATOP THE ARGUMENT
\rule{-1ex}{0ex}
\end{array}
}


\usetheme{CUDenver}
\renewcommand{\familydefault}{\sfdefault}
\usefonttheme[onlymath]{serif}
\usepackage{tikz}
\usetikzlibrary{arrows}
\setbeamertemplate{itemize items}{>>}
\setbeamertemplate{itemize subitem}[square]
\setbeamertemplate{itemize subsubitem}[ball]
\newcommand{\norm}[1]{\left\lVert#1\right\rVert}



\author{Jordan Hall}
\title[CUDenver Theme]{Thesis Proposal}
\institute[UCD]{
Department of Mathematical and Statistical Sciences\\
University of Colorado Denver
}
\date{Tuesday, December 4, 2018}

\begin{document}
% ------------------------------------------------
\begin{frame}[t,plain]
    \titlepage
\end{frame}

% start the content of the presentation

\begin{frame}{Overview}
\tableofcontents
\end{frame}


% ------------------------------------------------
\section{Literature Review and Framework}
% ------------------------------------------------
\begin{frame}

\begin{center}
\textbf{Notation}
\end{center}

\begin{itemize}

	\item We define a parameter space $\Lambda$ with dimension $N$ and a data space $\mathcal{D}$ with dimension $M$. Generally $M \leq N$.
	
	\item We define a parameter-to-data map $f$, which is noisy and $\nabla f$ may be inaccessible.
	
	\item We consider additive noise in $f$, which is modeled by draws $f(\lambda)=f(\lambda)+\epsilon(\lambda)$ for additive noise and $f(\lambda)=f(\lambda)(1+\epsilon(\lambda))$ for multiplicative noise.

\end{itemize}

\end{frame}

% ------------------------------------------------
\subsection{Data-Consistent Inversion}
% ------------------------------------------------

\begin{frame}

\begin{itemize}

	\item We will follow



\end{itemize}



\end{frame}






% ------------------------------------------------
\section{Research Questions}
% ------------------------------------------------
\begin{frame}

$x$

\end{frame}

\section{Preliminary Results and Research Plan}
% ------------------------------------------------
\begin{frame}

$x$

\end{frame}


\section{Timeline}
% ------------------------------------------------
\begin{frame}

$x$

\end{frame}


\section{References}
% ------------------------------------------------
\begin{frame}

$x$

\end{frame}







\end{document}

